\documentclass{report}
% ######################################### %
% # DO REMOVE THESE REPORTS OUT WHEN DONE # %
% ######################################### %
\usepackage[
	a4paper,
	left=1.5in,
	top=1in,
	right=1in,
	bottom=1in
]{geometry}

\usepackage{ownstyle}
\usepackage{titlesec}
\titleformat{\chapter}[display]{\centering\Large\bfseries}{บทที่~\thechapter}{6pt}{\centering\Large\bfseries}

\setmainfont{TH Sarabun New}
\setsansfont{TH Sarabun New}
\renewfontfamily{\thaifont}{TH Sarabun New}

\begin{document}

%\chapter{รายละเอียดการดำเนินงาน}

%\section{ภาพรวมของโครงงาน}

%\section{ส่วนประกอบโครงงาน}
%\subsection{Database}
%ในโครงงานนี้ มีการใช้ระบบฐานข้อมูล InfluxDB ในการเก็บข้อมูลจุดความร้อนที่ได้มาจาก FIRMS เพื่อใช้
%ในการประเมินหาบริเวณที่เป็นไฟป่า
%
%\subsection{Web Application}
%\subsection{Fire Prediction}
%\subsection{Navigation}
%\subsection{LINE Alert}

\chapter{ความก้าวหน้าการดำเนินงาน}

\section{Database}
\subsection{รายละเอียดการพัฒนา}
ในการพัฒนา ได้มีการสร้างฐานข้อมูลเพื่อใช้ในการเก็บข้อมูลจุดความร้อน
โดยใช้ InfluxDB เป็นระบบฐานข้อมูลสำหรับงานนี้ โดยทำการติดตั้งแบบส่วนตัวสำหรับการใช้งานกับระบบ%
นี้เท่านั้น

ข้อมูลที่ทำการจัดเก็บประกอบด้วยจุดความร้อนดิบที่อยู่ในประเทศไทยและพื้นที่ใกล้เคียงที่นำมาจาก FIRMS

\subsection{อุปสรรคในการพัฒนา}
ในระหว่างการพัฒนา ได้พบปัญหาในการทำการร้องขอข้อมูลจากระบบฐานข้อมูล เนื่องจาก InfluxDB 
มีภาษาที่ใช้ในการร้องขอข้อมูลได้สองแบบ 
โดยแบบแรกเป็น InfluxQL ซึ่งมีลักษณะคล้ายคลึงกับภาษาประเภท SQL และอีกแบบคือภาษา Flux
ซึ่งเป็นภาษาพิเศษที่สร้างขึ้นมาเพื่อใช้กับการประมวลผลและร้องขอข้อมูลจาก InfluxDB โดยเฉพาะ
ซึ่งข้อดีในการใช้ InfluxQL นั้นคือเรียบง่าย มีลักษณะคล้ายคลึงกับภาษากลุ่ม SQL ทำให้สามารถ
ทำการร้องขอข้อมูลได้เร็วกว่า แต่การใช้ Flux ในการร้องขอข้อมูลนั้นสามารถทำการประมวลผลข้อมูล%
เบื้องต้นได้ที่ระบบฐานข้อมูลเลย ทำให้สามารถทำการร้องขอข้อมูลที่เกี่ยวข้องได้แม่นยำขึ้น ลดการประมวลผล%
ที่ไม่เกี่ยวข้องกับงานประมวลผลหลักได้ แต่นั่นก็ทำให้การเขียนคำสั่งร้องขอข้อมูลนั้นมีความยาวและซับซ้อน%
เพิ่มขึ้นอย่างมาก

\subsection{แนวทางแก้ไขปัญหา}
ทำการประเมินว่าการประมวลผลข้อมูลเบื้องต้นที่ใช้นั้น สามารถนำมาประมวลผลที่ด้านผู้รับข้อมูลแทนได้หรือไม่
ดูว่าการประมวลผลด้านระบบฐานข้อมูลหรือด้านผู้รับนั้น แบบไหนที่ได้ประสิทธิภาพโดยรวมได้ดีกว่า 
แบบไหนมีความซับซ้อนน้อยกว่า และความซับซ้อนต่อผลที่ได้นั้นคุ้มค่าที่จะใช้หรือไม่ และทำการเลือกวิธีืที่%
เหมาะสมกับงานต่อไป

\section{FIRMS NRT Data Acquisition and Preprocessing}
\subsection{รายละเอียดการดำเนินงาน}
ในโครงงานนี้ มีการใช้ข้อมูลจุดความร้อนจาก FIRMS เป็นข้อมูลตั้งต้นในการทำนายหาพื้นที่ที่เป็นไฟป่าและ%
การทำงานอื่นๆที่เกี่ยวข้อง แต่ก่อนหน้าที่จะนำข้อมูลมาใช้นั้น ต้องมีการประมวลผลข้อมูลเบื่องต้นเพื่อนำข้อมูล%
ที่ไม่เกี่ยวข้องออกไปเสียก่อน

ได้ทำการพัฒนาคำสั่งที่สามารถใช้ในการร้องขอข้อมูล Near Real Time (NRT) 
โดยสามารถเลือกดาวเทียมเป็นต้นทางของข้อมูลจุดความร้อนได้
พร้อมทั้งวิธีการประมวลผลข้อมูลเบื้องต้นเพื่อให้ได้ข้อมูลที่เกี่ยวข้อง อย่างเช่นการสร้าง Timestamp
ของจุดข้อมูลที่ถูกต้อง และการกรองข้อมูลให้เหลือข้อมูลที่อยู่ภายในขอบเขต Bounding Box 
ของประเทศไทยเท่านั้น เนื่องจากข้อมูลที่ได้มาจากการร้องขอนั้นมีขอบเขตครอบคลุมทั้งพื้นที่ทวีบเอเชีย
ทางตะวันออกเฉียงใต้ ซึ่งเยอะเกินต้องการ
พร้อมทั้งการนำข้อมูลไปจัดเก็บลงระบบฐานข้อมูลเนื่องจากข้อจำกัดของลักษณะของข้อมูลที่ใช้ในระบบฐานข้อมูล

\subsection{อุปสรรคในการพัฒนา}
อุปสรรคในการพัฒนาส่วนนี้ ได้แก่การที่ข้อมูลบางส่วนนั้นมีประเภทของข้อมูลไม่ตรงกัน อย่างเช่นข้อมูล%
ค่าความมั่นใจในจุดความร้อนที่วัดได้ (confidence) ในข้อมูลประเภท NRT นั้นจะมีการจัดกลุ่มข้อมูล%
เป็นประเภทชัดเจนโดยแสดงผ่านตัวอักษร แต่ในข้อมูลที่เป็นข้อมูลเก่านั้น ค่าที่ได้จะเป็นค่าตัวเลข 
ซึ่งเมื่อทำการจัดเก็บลงฐานข้อมูลจะเกิดการขัดแย้งกันของประเภทข้อมูล ทำให้ไม่สามารถจัดเก็บข้อมูลนั้นได้

\subsection{แนวทางแก้ไขปัญหา}
แนวทางแก้ไขปัญหาที่คาดไว้ คือการกำหนดลักษณะและประเภทค่าของข้อมูลให้ชัดเจน โดยถ้าหากพบค่าที่%
มีประเภทค่าไม่ตรงกัน ให้ทำการแปลงข้อมูลจากประเภทหนึ่งไปยังอีกประเภทหนึ่งทั้งหมด โดยใช้วิธีการ%
ต่างๆเช่นการจัดกลุ่มค่าเป็นประเภทเพื่อให้มีลักษณะเหมือนกันทั้งหมด ก่อนทำการจัดเก็บต่อไป

\section{Web Application}

\section{Clustering}
\subsection{รายละเอียดการดำเนินงาน}
หลังจากได้ข้อมูลมาแล้ว การประมวลผลหลักแรกที่ทำคือการพยายามทำการจับกลุ่มจุดความร้อน เพื่อหาว่า%
บริเวณไหนที่มีจุดความร้อนอยู่ในบริเวณใกล้เคียงในปริมาณมากบ้าง เพื่อประกอบการระบุบริเวณที่คาดว่า%
จะมีไฟป่าเกิดขึ้น

ได้ทำการพัฒนากระบวนการจัดกลุ่มข้อมูลโดยใช้ DBSCAN เป็นวิธีในการจัดกลุ่ม โดยวิธีนี้สามารถทำการ%
จัดกลุ่มจุดที่อยู่ใกล่เคียงกันได้โดยที่ไม่ต้องระบุจำนวนกลุ่มของจุดข้อมูลที่มีอยู่ในข้อมูลทั้งหมด
โดยประเมินจากการมีของจุดข้อมูลอื่นภายในรัศมีที่ระบุจากจุดหนึ่งๆ

\subsection{อุปสรรคในการพัฒนา}
การเลือกรัศมีที่เหมาะสมในการจัดกลุ่มข้อมูล เพื่อให้สามารถจัดกลุ่มข้อมูลเพื่อหาพื้นที่ไฟป่าได้ตรงกับพื้นที่%
ที่เกิดไฟป่าขึ้นจริงที่สุด

\subsection{แนวทางแก้ไขปัญหา}

\section{Region of Interest}
\subsection{รายละเอียดการดำเนินงาน}

\subsection{อุปสรรคในการพัฒนา}

\subsection{แนวทางแก้ไขปัญหา}


\end{document}